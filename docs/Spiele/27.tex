\section{Trampolin}\label{trampolin}

Erforderliche Kompetenzen: Bewegung, Schleifen Gewonnene Kompetenzen:
Tastendrücke, Kostüme

\begin{figure}[ht]
    \centering 
    \includeimage{img/27.png}
    \caption[\Sectionname]{\Sectionname}
\end{figure}

\subsection{Beschreibung}\label{beschreibung}

In diesem Programm springt Scratch (die Katze) auf einem Trampolin und
kann in der Luft Figuren wie Drehungen und Salti ausführen.

Durch Gedrückthalten der Leertaste springt Scratch höher. Auf die Tasten
„a" und „s" reagiert Scratch mit Salti und auf die Taste „d" mit einer
Drehung.

\subsection{Durchführung}\label{durchfuxfchrung}

\subsubsection{Phase 1: Planung}\label{phase-1-planung}

Während der Planungsphase soll mit den Schülern folgende Schritte
vollzogen werden:

\begin{itemize}
\item
  Testen des fertigen Programmes durch die Schüler
\item
  Erläuterung des Blocks ``wird Taste * gedrückt''
\item
  Ideensammlung zu den Funktionen des Programms: Drehung und Salto der
  Katze; auch weitere Funktionen möglich !
\end{itemize}

\subsubsection{Phase 2: Vorbereitungen}\label{phase-2-vorbereitungen}

Auswählen des Hintergrundes.

\subsubsection{Phase 3: Programmierung}\label{phase-3-programmierung}

\begin{enumerate}
\item
  Aktionen der Figur wie z.B. Salti
\item
  Ggf. erstellen weiterer Kostüme für Kunststücke
\end{enumerate}

\subsubsection{Detaillierte
Programmbeschreibung}\label{detaillierte-programmbeschreibung}

Zunächst wird die Physik des Spiels (das Fallen und Abprallen am
Trampolin) programmiert und in den unteren Teil des Skripts verschoben,
und soll vom Schüler ignoriert werden. Dies ist in der Vorlage
enthalten.

Zum Programmieren der Physik ist die Variable Geschwindigkeit nötig. Um
ihren Wert wird die y-Position von Scratch kontinuierlich geändert. Ist
Geschwindigkeit größer Null so steigt Scratch nach oben, ist die
Geschwindigkeit kleiner als Null, so fällt Scratch. Wenn nun die
Geschwindigkeit ebenfalls kontinuierlich verringert wird, so entsteht
der Eindruch eines freien Falles. Dazu wird die Variable
Geschwindigkeit, bei jedem Durchlauf der Hauptschleife im Block ``Bewege
dich'' verringert und die y Position von Scratch um den Wert der
Geschwindigkeit verändert.

Zusätzlich wird in der Hauptschleife abgefragt, ob das Trampolin
berrührt wird. Ist dies der Fall so muss die Geschwindigkeit ihr
Vorzeichen ändern, was mit der Addition der Geschwindigkeit um minus
zweimal der Geschwindigkeit geschieht.

(Geschwindigkeit = Geschwindigkeit -2*Geschwindigkeit)

Wird zum Zeitpunkt der Berührung des Trampolins auch die Leertaste
gedrückt, so wird mithilfe des Blocks ``Springe hoch um 5'' die nun
positive Geschwindigkeit um fünf erhöht.

Außerdem wird bei jeder Berührung des Trampolins der
Geschwindigkeitsbetrag um eins verringert, sodass Scratch irgendwann
nicht mehr springt, was den Eindruck von Luftreibung hervorruft. An der
Stelle kann man die Schüler auch fragen, was Luftreibung bedeutet.

\subsubsection{Phase 4: Testen und
vorstellen}\label{phase-4-testen-und-vorstellen}

In der letzten Phase, die auch zweimal durchgeführt werden kann, müssen
alle (bei großen Gruppen: ausgewählte) Schülerinnen und Schüler ihr
Projekt vorstellen. Hier bietet es sich an, wenn eine dritte Person das
Programm bedienen muss und nicht die Programmiererin oder der
Programmierer. Der restliche Kurs schaut bei der Vorstellung zu und gibt
hinterher eine Rückmeldung. Für jedes Projekt kann so analysiert werden,
was gut und was weniger gut funktioniert und noch verbessert werden
muss. Wichtig vor allem bei jüngeren Klassenstufen: Die guten Teile des
Programmes besonders hervorheben!

\subsection{Erweiterungen}\label{erweiterungen}

\begin{itemize}
\tightlist
\item
  \emph{Mehrspielermodus}: Eine schnelle Schülerin könnte noch eine
  zweite Figur ins Spiel einbauen, welche mit anderen Tasten gesteuert
  werden kann. Dann können zwei Spielerinnen in demselben Spiel sich
  darum konkurrieren, wer die besseren Tricks ausführen kann.
\end{itemize}
