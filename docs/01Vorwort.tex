\section*{Vorwort}
Informatik und Programmierung ist ein spannendes Fachgebiet, das eine hohe Bedeutung in unserem Alltag hat. Informatik und Programmierung ist überall gegenwärtig und wird ständig genutzt - dennoch ist das Angebot für informatische Bildung an Schulen in Deutschland oftmals kaum oder gar nicht vorhanden.

An vielen Schulen fehlen die Kapazitäten, sei es personell, sei es finanziell - dabei ist Informatik und Programmierung als spannendes Ergänzungsfach schon ab der dritten Klasse denkbar.

Ziel dieses Buchs soll sein, eine Basis für die Durchführung von Informatik-Kursen für Kinder im Grundschulalter und in der frühen Sekundarstufe~I zu bieten. Auf diese Weise können Sie als Lehrperson auch ohne tiefgreifende Informatikkenntnisse schnell Anregungen finden, wie Schülerinnen und Schüler an dieses spannende Feld herangeführt werden können.

Die Struktur des Buches ist bewusst offen gehalten und bietet eine Basis für einen sehr individuellen, auf Schülerinteressen angepassten Unterricht. Inhalt sind Projekte unterschiedlicher Themengebiete, die den Schülerinnen und Schülern auf vielfältige Weise die Grundkonzepte des Programmierens näher bringen. Mit steigender Lernkurve kann auch die Komplexität der Projekte schrittweise erhöht werden.

Das vorliegende Konzept wurde in einem einjährigen Modellversuch an einer Grundschule in Aachen im Jahr 2013/14 getestet und im Schuljahr 2014/15 an über zehn weiteren Schulen erfolgreich erprobt.

Unser Ziel ist es, Informatik flächendeckend an jeder Schule zu einem festen Bestandteil im Kursangebot zu machen. Die gewählten Projekte erfordern minimalen Einarbeitungsaufwand für Sie als Betreuerin oder Betreuer, werden aber zugleich alle denkbaren Schülerinteressen abdecken können. Überdies sind wir stets bemüht, die Inhalte des Modulhandbuchs weiter auszubauen.
\newpage
In diesem Sinne hoffe ich, dass dieses Werk Ihnen als Lehrperson eine Hilfe ist, einen spannenden Kurs vorzubereiten und den Schülerinnen und Schülern an Ihrer Schule die Möglichkeit zu bieten, einen erfolgreichen Einstieg in die Welt der Informatik zu finden.

\ \\Mit freundlichen Grüßen \\

\ \\Steffen Schneider\\
Aachen, den \today
