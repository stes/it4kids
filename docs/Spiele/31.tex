\section{Aquarium}\label{aquarium}

Erforderliche Kompetenzen: Bewegungen, Fühlen

Erworbene Kompetenzen: Programmieren mehrerer Figuren, Operatoren

\begin{figure}[ht]
    \centering 
    \includeimage{img/31.png}
    \caption[\Sectionname]{\Sectionname}
\end{figure}

\subsection{Beschreibung}\label{beschreibung}

Dieses Projekt eignet sich sehr gut zum Einstieg in die
Spieleentwicklung. Ziel des Spiels ist es, eine steuerbare Figur an
anderen computergesteuerten Figuren vorbeizubewegen, ohne diese zu
berühren.\\
In dem Aquariums befinden sich mehrere Fische, die es fortlaufend
durchqueren. Gleichzeitig befindet sich eine vom Spieler steuerbare
Figur in dem Aquarium, die zum oberen Rand des Spielfeldes bewegt werden
soll, ohne einen Fisch zu berühren. Wird ein Fisch berührt, so sinkt die
Figur zurück auf die Startposition.

\subsection{Durchführung}\label{durchfuxfchrung}

\subsubsection{Phase 1: Planung}\label{phase-1-planung}

Während der Planungsphase sollten mit den Schülerinnen und Schülern die
folgenden Schritte vollzogen werden:

\begin{itemize}
\item
  Erläuterung des Projektes
\item
  Überlegungen und Ideensammlung für Hintergrund, Figuren, etc.
\item
  Überlegungen zur Programmierung und der nötigen Aufgaben:

  \begin{itemize}
  \item
    Bewegung der Fische
  \item
    Bewegung der Spielfigur/Steuerung
  \item
    Erkennen der Kollision zwischen Fisch und Figur
  \end{itemize}
\item
  Sammeln aller Ideen und Erstellung der TODO Liste
\end{itemize}

Anschließend kann mit der Umsetzung des Projekts begonnen werden.

\subsubsection{Phase 2: Vorbereitungen}\label{phase-2-vorbereitungen}

Zunächst sollte der Hintergrund gezeichnet werden. Für die Figuren
bieten sich diverse Fische, sowie Taucher, Seesterne oder Oktopusse aus
der Scratch-Bibliothek an. Die Gestaltung des Hintergrunds ist den
Schülerinnen und Schülern überlassen.

\subsubsection{Phase 3: Programmierung}\label{phase-3-programmierung}

Im Anschluss an die vorbereitenden Maßnahmen kann mit der Programmierung
des Spiels begonnen werden. Als Vorschlag sei die folgende Reihenfolge
genannt, in der Schritt für Schritt leicht validierbare Teile des Spiels
entstehen:

\begin{enumerate}
\item
  \emph{Bewegung der Fische:} Die Fische folgen einem linearen
  Bewegungspfad, der sie beim anstoßen an den Rand wieder am anderen
  erscheinen lässt. Wichtig ist eine fortlaufende Schleife.
\item
  \emph{Steuerung der Spielfigur:} Die Figur lässt sich am besten über
  die Pfeiltasten steuern, wobei jede einer simplen Bewegung in die
  entsprechende Richtung zugeordnet wird. Hierbei lassen sich die
  Unterschiede zwischen den ``gehe'' und ``ändere x/y'' Befehlen
  erklären.
\item
  \emph{Kollision} Es muss ein Erkennen verschiedener Figuren erfolgen
  können. Hierzu bietet sich die Einführung des ``oder'' Operators an.
  Nach der Kollision gleitet die Spielfigur zurück an ihren Startpunkt.
  Hierbei ist ein Kostümwechsel möglich.
\end{enumerate}

\subsubsection{Phase 4: Testen und
vorstellen}\label{phase-4-testen-und-vorstellen}

In der letzten Phase, die auch zweimal durchgeführt werden kann, müssen
alle (bei großen Gruppen: ausgewählte) Schülerinnen und Schüler ihr
Projekt vorstellen. Hier bietet es sich an, wenn eine dritte Person das
Programm bedienen muss und nicht die Programmiererin oder der
Programmierer. Der restliche Kurs schaut bei der Vorstellung zu und gibt
hinterher eine Rückmeldung. Für jedes Projekt kann so analysiert werden,
was gut und was weniger gut funktioniert und noch verbessert werden
muss. Wichtig vor allem bei jüngeren Klassenstufen: Die guten Teile des
Programmes besonders hervorheben!

Es kann anschließend eine weitere Programmierphase angestoßen werden,
bei der jedem die Chance gegeben wird, die vorhandenen Mängel noch zu
beheben.

\subsection{Erweiterungen}\label{erweiterungen}

Mögliche Erweiterungen sind z.B.:

\begin{itemize}
\item
  Einführung eines Ziels an der Wasseroberfläche
\item
  Hinzufügen von statischen Hindernissen
\item
  Kostümwechsel der Fische bei Kollision (erfordert
  Firgurenkommunikation)
\end{itemize}
