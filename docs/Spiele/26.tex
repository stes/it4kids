\section{Pacman}\label{pacman}

Erforderliche Kompetenzen: BEWEGUNGEN, FÜHLEN, BEDINGUNGEN\\
Erworbene Kompetenzen: Fortgeschritten

\begin{figure}[ht]
    \centering 
    \includeimage{img/26.png}
    \caption[\Sectionname]{\Sectionname}
\end{figure}

\subsection{Beschreibung}\label{beschreibung}

Dieses Projekt ist deutlich fortgeschrittener. Es geht um die
Programmierung des bekannten Spiels ``Pacman''. Die Schülerinnen und
Schüler haben hier die Möglichkeit, von der Planungsphase des Projekts
über das Zeichnen von Hintergrund und Gestaltung der Figur alle
wesentlichen Abläufe der Spieleprogrammierung kennenzulernen.

Es ist besonders bei diesem Projekt sehr wichtig, dass die Schülerinnen
und Schüler zu jedem Zeitpunkt sehr klare Vorstellungen über ihre
nächste Aufgabe haben und nicht tatenlos am PC sitzen. Daher gibt es
einige Erweiterungen für besonders schnelle Schülerinnen und Schüler.

\subsection{Durchführung}\label{durchfuxfchrung}

Im Folgenden ist die Durchführung des Projekts geschildert. Sollte es
triftige Gründe geben, kann die Lehrperson gerne auch ein alternatives
Konzept heranziehen.

\subsubsection{Phase 1: Planung}\label{phase-1-planung}

Die Planungsphase wird in den ganzen Gruppe vollzogen. Die Lehrperson
erläutert das Projekt und die Ziele der aktuellen Stunde werden geklärt.
Beim Projekt Pacman müssen konkret geplant werden:

\begin{itemize}
\item
  Präsentation eines komplett fertigen Spiels
\item
  Überlegungen und Ideensammlung für Hintergrund, Figur, etc.
\item
  Überlegungen zur Programmierung und der nötigen Aufgaben:

  \begin{itemize}
  \item
    Bewegungen der Spielfigur auf Tastendruck: Umsetzung durch
    entsprechende Ereignisse
  \item
    Anstoßen der Spielfigur an die Ränder des Spielfelds
  \item
    Anstoßen der Spielfigur an die Mauern
  \item
    ``Essen'' der Spielfigur
  \item
    Punktezähler und dessen Funktionsweise
  \end{itemize}
\item
  Nach dem Sammeln der Ideen bestimmen einer sinnvollen Reihenfolge,
  diese TODO Liste könnte z.B. an die Tafel geschrieben werden.
\end{itemize}

Mit der nun entstandenen Aufgabenliste können die Schüler mit der
eigentlichen Umsetzung des Projekts beginnen.

\subsubsection{Phase 2: Objekte und Hintergrund
zeichnen}\label{phase-2-objekte-und-hintergrund-zeichnen}

Es bietet sich auch aus Zeitgründen an, im Anschluss an die Planung mit
dem Zeichnen von Hintergründen, Figuren etc. die erste Stunde
abzuschließen, da dies relativ leicht umsetzbare Aufgaben sind.
Besonders schnelle Schülerinnen und Schüler können anschließend direkt
mit der Programmierung beginnen.

\subsubsection{Phase 3: Programmieren}\label{phase-3-programmieren}

Hier werden die zuvor besprochenen Aufgaben in Programmcode umgesetzt.
Es macht Sinn, dabei die folgende Reihenfolge einzuhalten:

\begin{enumerate}
\item
  \emph{Steuerung der Figur:} Zuerst programmieren die Schülerinnen und
  Schüler die Bewegung der Spielfigur. Die Figur soll sich dabei
  konstant in Vorwärtsrichtung bewegen (in einer Endlosschleife, dem
  \emph{Game Loop}). Auf Tastendruck (wird über Steuerungsblöcke
  abgegriffen) findet eine Drehung statt. Es können die Tasten A, S, D,
  F oder das Steuerkreuz verwendet werden.
\item
  \emph{Anstoßen der Spielfigur an den Rand:} Beim Anstoßen an den
  Spielfeldrand soll sich die Figur nicht weiterbewegen
\item
  \emph{Anstoßen der Spielfigur an die Wände:} Wenn die Spielfigur die
  Wände des Spielfelds (gezeichnet in einer möglichst einmalig
  vorhandenen Farbe) berührt, soll das Spiel als verloren gezählt
  werden.
\item
  \emph{Essen} \ldots{}
\item
  \emph{Punktezähler:} Der Punktezähler wird über eine Variable
  realisiert. Die Variable muss angelegt und benannt werden und
  anschließend im Modus \emph{Groß} auf dem Spielfeld angezeigt werden.
  Bei Neustart des Spiels wird die Variable auf null gesetzt, bei dem
  Essen eines Elements wird sie entsprechend erhöht.
\end{enumerate}

\subsubsection{Phase 4: Testen und
vorstellen}\label{phase-4-testen-und-vorstellen}

In der letzten Phase, die auch zweimal durchgeführt werden kann, müssen
alle (bei großen Gruppen: ausgewählte) Schülerinnen und Schüler ihr
Projekt vorstellen. Hier bietet es sich an, wenn eine dritte Person das
Programm bedienen muss und nicht die Programmiererin oder der
Programmierer. Der restliche Kurs schaut bei der Vorstellung zu und gibt
hinterher eine Rückmeldung. Für jedes Projekt kann so analysiert werden,
was gut und was weniger gut funktioniert und noch verbessert werden
muss. Wichtig vor allem bei jüngeren Klassenstufen: Die guten Teile des
Programmes besonders hervorheben!
