\chapter {Informatik mit Stift und Papier}

%Für den optimalen Einsatz dieses Lehrbuchs im Unterricht empfehle ich den Einsatz des Scratchsticks, der in diesem Kapitel unter anderem behandelt wird. Der Vorteil ist die Ausnutzung aller Unterrichtsmethoden, die den Schülerinnen und Schülern einen maximalen Lernerfolg bieten. Alternativ kann, wie oben schon beschrieben, auch auf den Scratch Editor (Version 1.4 offline, Version 2 online oder Version 2 Beta online) oder den Snap Editor (online und offline) zurückgegriffen werden.

In den ersten ein bis zwei Unterrichtsstunden sollte den Schülern, insbesondere wenn diese noch keinerlei Zugang zur Informatik hatten, kurz dargelegt werden, was Informatik eigentlich ist und mit welchen Mitteln im Unterricht gearbeitet wird. Dazu kann die Lehrperson kreativ werden oder aber auch eine Einführung aus diesem Buch verwenden.

\section{Zauberwürfel}\label{zauberwuxfcrfel}

Materialien: Ein Zauberwürfel(``Rubicks Cube'') Schwierigkeit: Eher für
ältere Schüler

Der Zauberwürfel sollte bis auf einige verbleibende Drehungen gelöst
sein, auf jeden Fall muss die Lösung relativ offensichtlich sein. Einem
Schüler werden die Augen verbunden, er erhält anschließend den Würfel,
den er vor sich auf den Tisch legt.

Anschließend haben die Kinder die Möglichkeit, nacheinander dem
Probanten Anweisungen zu geben. Dazu kann das vorliegende Spektrum an
``Befehlen'' genutzt werden:

\begin{itemize}
\item
  Drehe nach oben: Der Schüler dreht den Würfel so, dass die von ihm
  weg gerichtete Seite nun in Richtung Decke zeigt
\item
  Drehe nach rechts: Der Schüler dreht den Würfel so, dass die von ihm
  nach links ausgerichtete Seite nun in Richtung Decke zeigt
\item
  Verändere: Der Schüler dreht die obere ``Scheibe'' des Würfels im
  Uhrzeigersinn
\end{itemize}

Die Befehle sind bewusst auf ein Minimum reduziert, gerade so, dass das
Problem lösbar ist. Während dem Schüler die Befehle mitgeteilt werden,
schreibt die Lehrperson oder ein weiterer Schüler die Befehle an die
Tafel. Nach einiger Zeit sollte das Problem gelöst sein, wenn vielleicht
auch nicht optimal. Dann wird der Würfel an einen weiteren Schüler
gegeben (wieder entsprechend präpariert), der die Augenbinde erhält. Ein
weiterer Schüler liest das Programm vor, das anschließend umgesetzt
wird.

Optional ist es möglich, die Schüler mit der Frage zu konfrontieren, wie
der erste Zustand wiederhergestellt werden kann und wie sich das Problem
noch effizienter lösen lässt.

%// TODO lege als Tabelle an!

\section{Labyrinth}\label{labyrinth}

Wie Zauberwürfel, nur mit einem Labyrinth mit Stift und Papier.
Mögliches Szenario: Zuerst sieht man das Labyrinth auf eine
Übersichtskarte, anschließend allerdings nur noch einen schwarzen
Hintergrund, trotzdem muss man den Ausgang finden

\section{Labyrinth im Klassenraum}\label{labyrinth-im-klassenraum}

Ein Schüler erhält eine Augenbinde, im Klassenraum wird ein kleiner
Parcours aufgebaut. Die anderen Schüler müssen durch Anweisungen aus dem
untenstehenden Befehlssatz versuchen, den Schüler sicher und ohne
anzuecken durch den Klassenraum zu geleiten.

\begin{itemize}
\item
  Schritt vor
\item
  Drehe nach rechts (Drehung um 90 Grad nach rechts)
\item
  Drehe nach links (Drehung um 90 Grad nach links)
\end{itemize}

Dieses Experiment kann sehr gut als Einstieg genutzt werden, um die
Unterschiede in der ``Genauigkeit'' von realer Welt und Computer zu
verdeutlichen. Das Programm wird hier beim zweiten Schüler vermutlich
aufgrund anderer Schrittweiten o. ä. nur noch ansatzweise funktionieren.
Anschließend kann das Experiment ``Programmieren mit Stift und Papier''
gewissermaßen als Kontrast verwendet werden.

\section{Labyrinth mit Folie und
Papier}\label{labyrinth-mit-folie-und-papier}

Es ist teilweise in den Klassenzimmern nicht möglich die Tische zu
verrücken um ein Labyrinth zu bauen. Der Grund dafür ist entweder, dass
die Tische am Boden fest montiert sind oder Teilweise ist auch die
Gruppengröße für das Klassenraumprojekt nicht angemessen. In diesem Fall
gibt es die Alternative für eine Übung in Kleingruppen am Arbeitsplatz.
Man dasselbe Experiment auch mit einem augedruckten Labyrinth in kleinen
Gruppen durchgeführt werde. Man bildet kleine Gruppen bis zu einer
Gruppengröße von bestelnfalls 3 bis maximal 5 Personen. Jede Gruppe
erhält eine Folie, ein Papier mit einem Raster und ein Papier, auf dem
das Labyrinth abgedruckt ist. Eine Person in der Gruppe darf das
Labyrinth sehen und der Zeichnenden Person anweisungen geben. Ein
Gruppenmitglied ist der Zeichner. Dieses Gruppenmitglied hat die Aufgabe
auf der Folie , die auf dem Rasterpapier liegt, die Anweisungen der
Person aufzuzeichnen, die das Labyrinth lösen muss. Eine dritte Person
schreibt die Befehle untereinander auf, die die Anweisende Person gibt.
So entsteht ein Programm.

Der Befehlssatz besteht hier aus:

\begin{itemize}
\item
  Ein Kästchen vor
\item
  Drehe links (um 90 Grad)
\item
  Drehe rechts (um 90 Grad)
\end{itemize}

Im Anschluss an diese Übung kann dasselbe Labyrinth mit Scratch gelöst
werden. Hier kann in Anlehnung an das in der Gruppe erstellte Programm
eine Maus durch das Labyrinth zum Käse geführt werden. Dieses Projekt
vereint das Projekt ``Maus zum Käse'' und ``Labyrinth''. Für Schüler,
die bereits sehr schnell das erste Programm in Scratch fertiggestellt
haben, liegt im Anhang ein anspruchsvolleres Labyrinth vor, um das
Erlernte zu vertiefen.

Die Parallelen zur allgemeinen Programmierung können mit diesem Projekt
sehr schön bildhaft dargestellt werden. Die Kinder sollen verstehen, wie
der Programmierer mit dem Computer und der Entwicklungsumgebung
arbeitet. Dafür werden zu Beginn drei Rollen verteilt:

\begin{itemize}
\item
  \emph{Programmierer} : Anweisende Person, dieser erhält das Labyrinth
  auf Folie. Der Programmierer kennt damit das zu lösende Problem, kann
  es aber nur mithilfe des Computers lösen
\item
  \emph{Computer}: Der ``Computer'' erhält ein leeres Blatt Papier mit
  einem vorgezeichneten Raster. Auf dieses wird Schritt für Schritt der
  Weg eingezeichnet, den der Programmierer vorsagt. Der Computer hat
  also keine Kenntnis über das Problem und befolgt lediglich die
  Anweisungen des Programmierers.
\item
  \emph{Entwicklungsumgebung/Sekretär}: Damit der Programmierer das
  Problem lösen kann, ohne bei jedem Durchlauf die Befehle neu zu
  überlegen, werden alle Anweisungen auf einem leeren Blatt Papier
  festgehalten. Wenn das Problem dann einmal fertig gelöst ist, kann mit
  diesen Anweisungen das Labyrinth immer wieder durchlaufen werden.
\end{itemize}

\subsection{Erweiterungen}\label{erweiterungen}

Wenn das Problem mit einfachen Anweisungen gelöst ist, kann als neuer
Befehl die ``Wiederhole''-Struktur eingeführt werden. Mit dem Befehl

Wiederhole (N Mal)

wird der darauffolgenden Befehl entsprechend wiederholt ausgeführt. Die
Schülerinnen und Schüler können dann mit diesem neuen Befehl ihr
Programm entsprechend vereinfachen.

Mit leichten Abstrichen lässt sich diese Vorgehensweise, wenn nötig,
auch auf einer Tafel durchführen.

\subsection{Überleitung}\label{uxfcberleitung}

Nach erfolgreichem Abschluss des Projekts kann die Arbeit am PC
beginnen. Empfohlenes Projekt für den Anschluss ist wahlweise ``Maus zum
Käse'' vor allem für jüngere Schülerinnen und Schüler aus der
Grundschule oder auch ``Labyrinth''.
