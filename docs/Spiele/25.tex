\section{Helikopter}\label{helikopter}

Erforderliche Kompetenzen:Bewegung, Bedingungen, Schleifen Gewonnene
Kompetenzen: Kostüme Aussehen, Zufall

\begin{figure}[ht]
    \centering 
    \includeimage{img/25.png}
    \caption[\Sectionname]{\Sectionname}
\end{figure}

\subsection{Beschreibung}\label{beschreibung}

Der Spieler steuert einen Helikopter durch einen Regen von Hindernissen,
welchen er ausweichen soll. Der Helikopter kann mit Pfeiltasten in alle
Richtungen auf dem ganzen Spielfeld bewegt werden. Die Hindernisse sind
bunt und ändern kontinuierlich die Farbe. Stößt der Helikopter mit einem
der Hindernisse zusammen, bricht er auseinander und das Spiel ist
vorbei.

\subsection{Durchführung}\label{durchfuxfchrung}

\subsubsection{Phase 1: Planung}\label{phase-1-planung}

In der Planungsphase werden mit den Schülern die folgenden Dinge
besprochen:

\begin{itemize}
\item
  Wie fügt man die gewünschte Figur aus der Scratch-Bibliothek in das
  Spiel hinzu. Wie fügt man ein zusätzliches Zerstörungskostüm hinzu? Wo
  kann man in der Scratch-Umgebung ein Quadrat (als Hindernis) zeichnen?
\item
  Wie programmiert man das Bewegen des Helikopters und das Spielende
  durch Zusammenstoß mit dem Hindernis?
\item
  Und die wichtigste Frage: Wie programmiert man die Bewegung des
  Quadrats (als Hindernis) und seine Vermehrung?
\end{itemize}

\subsubsection{Phase 2: Vorbereitungen}\label{phase-2-vorbereitungen}

In der Vorbereitungsphase zeichnen die Schüler das Hindernis (ein
Quadrat oder Ähnliches) oder wählen eines aus der Scratch-Bibliothek
aus. Außerdem fügen sie dem Helikopter (oder einer anderen gewählten
Figur wie z.B. ein Flugzeug) ein Zerstörungskostüm hinzu.

\subsubsection{Phase 3: Programmierung}\label{phase-3-programmierung}

\begin{enumerate}
\item
  Bewegung des Helikopters mit den Pfeiltasten
\item
  Helikopter fühlt den Zusammenstoß mit dem Hindernis, worauf das Spiel
  endet
\item
  Ein Quadrat nach dem anderen taucht rechts irgendwo zufällig auf und
  fliegt nach links
\item
  Jedes Quadrat ändert kontinuierlich seine Farbe.
\end{enumerate}

\subsubsection{Detailierte
Programmbeschreibung}\label{detailierte-programmbeschreibung}

Das Zerstörungskostüm für den Helikopter kann man erstellen, indem man
in der Scratchumgebung den Helikopter aus der Scratch-Bibliothek lädt,
beim Helikopter auf Kostüme klickt, das aktuelle Kostüm per Rechtsklick
dupliziert und das neu erzeugte Kostüm modifiziert. Man verändert es,
indem man mit dem Auswahl-Tool einige Teile des Helikopters auswählt und
diese auseinander zieht.

Wenn die Flagge angeklickt wird erzeugt die Figur ``Hindernis'' mithilfe
eines ``wiederhole fortlaufend''-Blocks jede Sekunde einen Klon von sich
selbst. Unter dem ``Wenn ich als Klon entstehe''-Block wird die
x-Koordinate fest auf 300 (hinter des rechten Rands) und die
y-Koordinate auf eine zufällige Zahl zwischen -180 und 180 gesetzt. Dann
tritt der Klon in eine Schleife, in welcher er sich nach links bewegt
und den Effekt seiner Farbe ändert. Die Schleife bricht ab, sobald die
x-Position kleiner als -200 ist. Danach löscht sich der Klon selber mit
dem Befehl ``lösche diesen Klon''.

Helikopter detektiert außerdem die Kollision mit dem Hindernis in einer
Endlosschleife, woraufhin der Helikopter zum Zerstörungskostüm wechselt.
Nach einer Sekunde wird das Spiel mit ``stop all'' beendet.

\subsubsection{Phase 4: Testen und
vorstellen}\label{phase-4-testen-und-vorstellen}

In der Vorstellungsphase kann ein beliebiger Schüler sein Spiel
vorstellen, falls er es möchte. Dabei kann man explizit auf die
einzelnen Teilprobleme hinweisen, welche der Schüler gelöst hat. Aber
man kann die anderen auch fragen, was er verbessern könnte, wenn er an
diesem Projekt nächste Stunde noch arbeiten will.
